\documentclass[12pt]{extarticle}
\usepackage[top=2cm,bottom=2cm,left=2.5cm,right=2cm]{geometry}
\usepackage[utf8]{inputenc}
\usepackage[T2A]{fontenc}
\usepackage[russian]{babel}
\renewcommand{\baselinestretch}{1.0}
\usepackage{amssymb}
\usepackage{amsmath}

\begin{document}

\textbf{Вариант 1} \\
1. Пользователем заданы вещественные положительные числа a, b, c. Выяснить, существует ли
треугольник со сторонами a, b, c. \\
2. Даны натуральные числа m и n. Вычислить $1^n + 2^n + \ldots + m^n$.
3. Алгоритм вычисления значения функции $F(n)$, где $n$ — натуральное число, задан следующими соотношениями:
$$\begin{cases}
	F(n) = n+4, n \leq 2 \\
	F(n) = F(n-1) + F(n-2), n > 2.
\end{cases}$$
В ответе приведите дерево рекурсивных вызовов, осуществите его обратный обход, результат вызова $F(7)$.

\textbf{Вариант 2} \\
1. Пользователем заданы вещественные положительные числа a, b, c. Если существует тре-
угольник со сторонами a, b, c, то определить, является ли он прямоугольным. \\
2. Составить программу для нахождения всех натуральных решений (x и y)
уравнения $x^2 + y^2 = k^2$, где $x$, $y$ и $k$ лежат в интервале от 1 до 30. Решения, ко-
торые получаются перестановкой x и y, считать совпадающими.\\
3. Алгоритм вычисления значения функции $F(n)$, где $n$ — натуральное число, задан следующими соотношениями:
$$\begin{cases}
	F(n) = 2, n <= 2 \\
	F(n) = F(n-1) + 3*F(n-1), n > 2.
\end{cases}$$
В ответе приведите дерево рекурсивных вызовов, осуществите его обратный обход, результат вызова $F(5)$.

\textbf{Вариант 3} \\
1. Пользователем заданы три вещественных числа. Используя только два неполных условных
оператора, определить максимальное и минимальное значение заданных чисел.\\
2. Дано натуральное число n. Вычислить $1^1 + 2^2 + \ldots + n^n$.
3. Алгоритм вычисления значения функции $F(n)$, где $n$ — натуральное число, задан следующими соотношениями:
$$\begin{cases}
	F(n) = n-5, n <= 2 \\
	F(n) = F(n-2) + 3*F(n-3), n > 2.
\end{cases}$$
В ответе приведите дерево рекурсивных вызовов, осуществите его обратный обход, результат вызова $F(8)$.

\textbf{Вариант 4} \\
1. Вывести на экран номер четверти координатной плоскости, которой принад-
лежит заданная пользователем точка с координатами (х, у), при условии. \\
2. Найти натуральное число из интервала от a до b с максимальной суммой де-
лителей.\\
3. Алгоритм вычисления значения функции $F(n)$, где $n$ — натуральное число, задан следующими соотношениями:
$$\begin{cases}
	F(n) = 3, n <= 2 \\
	F(n) = F(n-4) + 2*F(n-1), n > 2.
\end{cases}$$
В ответе приведите дерево рекурсивных вызовов, осуществите его обратный обход, результат вызова $F(7)$.

\end{document}